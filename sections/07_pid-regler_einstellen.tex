
\section{Einstellen eines PID-Reglers} 

\subsection{Pol-Nullstellenkürzung}{164}

Hierbei handelt es sich um eine \textbf{analytische} Einstellmetode \textrightarrow\ LZI-Modell der Strecke muss vorliegen!
Der Regler wird dann so entworfen, dass er die \textbf{invertierte Stecke} enthält.
\begin{align*}
    G_R(s) &= \frac{K_R}{s} G_s^{-1}(s) \\
    G_0(s) &= G_R(s) \cdot G_S(s) \frac{K_R}{s} G_s^{-1}(s) \cdot G_S(s) = \frac{K_R}{s} \text{ \textrightarrow\ Integrator} \\
    G_f(s) &= \frac{G_0(s)}{1 + G_0(s)} = \frac{1}{\frac{1}{K_R} s + 1} \text{ \textrightarrow\ PT}_1 \text{-System} 
\end{align*}
Für die Übertragungsfunktionen gelten die folgenden Bezeichnungen:

\begin{minipage}[t]{0.48\columnwidth}
    \begin{tabular}{ll}
    $G_R(s)$    & UTF Regler \\
    $G_S(s)$    & UTF Stecke
\end{tabular}
\end{minipage}
\hfill
\begin{minipage}[t]{0.48\columnwidth}
    \begin{tabular}{ll}
        $G_0(s)$    & UTF offener Regelkreis \\
        $G_f(s)$    & UTF geschlossener Regelkreis  
    \end{tabular}
\end{minipage}


\subsubsection{Eigenschaften der Pol-Nullstellenkürzung}

\begin{outline}
    \1 \textbf{Pol-Nullstellenkürzung darf nur in der linken komplexen Halbebene durchgeführt werden!}
    % \1 Offener Regelkreis $G_0(s)$ wird zu einem Integrator \textrightarrow\ Optimalfall!
    % \1 Geschlossener Regelkreis $G_f(s)$ wird zu einem $\text{PT}_1$-System \textrightarrow\ Optimalfall!
    \1 Das Konzept funktioniert nicht immer
        \2 Inverse $G_s^{-1}(s)$ kann sehr sensitiv auf Modellparameter sein \\
            \textrightarrow\ braucht sehr genaues Modell
        \2 Instabile Stecken
        \2 Stecken mit Verzögerungen bzw. Totzeiten (\textrightarrow\ Regler wird akausal)
\end{outline}


\subsection{Empirische Einstellregeln}

\textbf{Idee:} Anhand weniger Messungen versucht man, über die Stecke genug Informationen zu gewinnen, um einen Regler entwerfen
zu können.


\subsubsection{Einstellung via Schrittantwort}{164-166}

\textbf{Idee:} Eine Stecke wird mit einem Eingangssignal $u(t) = A \cdot \varepsilon(t)$ angeregt und ihre 
Schrittantwort $y(t)$ wird gemessen. An diese gemessene Schrittantwort $y(t)$ wird ein \textbf{$\text{PT}_1$-System mit Totzeit}
'gefittet'. Die daraus entstehenden Parameter werden für die Regler-Dimensionierung verwendet.
$$ \text{PT}_1 \text{-System mit Totzeit} \quad G_0(s) = \frac{K_s}{s \cdot T_g + 1} e^{- s T_u} $$

\begin{minipage}[c]{0.45\columnwidth}
    \includegraphics[width=\columnwidth]{images/pid_regler_empirisch_einstellen.png}
\end{minipage}
\hfill
\begin{minipage}[c]{0.52\columnwidth}
    \begin{center}
        \textbf{\myul{Vorgehen $\text{PT}_1$ fitten / Parameter bestimmen}}
    \end{center}

    \begin{outline}
        \1 Tangente an Wendepunkt $Q$ einzeichnen
        \1 Parameter $T_u$, $T_g$ und $K_s$ gemäss Grafik bestimmen
            \2 $K_s$: Verstärkung
            \2 $T_u$: Verzugszeit
            \2 $T_g$: Ausgleichszeit
        \1 Konstanten für Tabellen bestimmen
            \2 $\mu = \frac{T_g}{T_u}$
            \2 $q = \frac{T_g}{T_u \cdot K_s} = \mu \cdot \frac{1}{K_s}$
    \end{outline}
\end{minipage}

\textbf{\myul{Regelbarkeit der Stecke}} \\
Gut regelbar heisst, die Zeitkonstante des geschlossenen Regelkreises ist kleiner als diejenige des offenen Regelkreises.

\begin{itemize}
    \item Gut regelbar: $\mu < 3$
    \item Schlecht regelbar: $\mu > 10$
\end{itemize}

\vspace{0.1cm}

\textbf{ACHTUNG: Struktur der Regler beachten!} Die Tabelle liefert Parameter für Regler in \textbf{serieller Form}
(siehe Abschnitt~\ref{PID-Regler multiplikativ}) 

\begin{tabular}{c c c }
    $ \boxed{ G_{\rm PID}(s) = K_R \cdot \Bigg( 1 + \frac{1}{s \cdot T_N} + s \cdot T_V \Bigg) }$ & 
    $ \boxed{ G_{\rm PI}(s) = K_R \cdot \Bigg( 1 + \frac{1}{s \cdot T_N} \Bigg) }$ &
    $ \boxed{ G_{\rm P}(s) = K_R }$ 
\end{tabular}

\begin{center}
    \begin{tabular}{|c | c | c | c | c|}
        \toprule
        Regler      & Methode       & $K_R$             & $T_N$                 & $T_V$             \\
        \midrule
                    & ZN            & $1.0 \cdot q$     & $-$                   & $-$               \\
        P-Regler    & CHR (20 \%)   & $0.7 \cdot q$     & $-$                   & $-$               \\
                    & CHR (0 \%)    & $0.3 \cdot q$     & $-$                   & $-$               \\
        \midrule
                    & ZN            & $0.9 \cdot q$     & $3.33 \cdot T_u$      & $-$               \\
        PI-Regler   & CHR (20 \%)   & $0.6 \cdot q$     & $1.0 \cdot T_g$       & $-$               \\
                    & CHR (0 \%)    & $0.35 \cdot q$    & $1.17 \cdot T_g$      & $-$               \\
        \midrule
                    & ZN            & $1.2 \cdot q$     & $2.0 \cdot T_u$       & $0.5 \cdot T_u$   \\
        PID-Regler  & CHR (20 \%)   & $0.6 \cdot q$     & $1.0 \cdot T_g$       & $0.47 \cdot T_u$  \\
                    & CHR (0 \%)    & $0.35 \cdot q$    & $1.17 \cdot T_g$      & $0.5 \cdot T_u$   \\
        \bottomrule
    \end{tabular}
\end{center}

\textbf{Hinweis:} Die Prozentwerte bei CHR beschreiben den Sollwert für Überschwinger.
Zu beachten ist, dass diese Werte durch die empirischen Einstellregeln nicht garantiert werden.


\subsubsection{Einstellung via Stabilitätsgrenze}{166-167}

\textbf{Idee:} Eine stabile Stecke wird mit \textbf{P-Regler} betrieben. Die Verstärkung $K_R$ des Reglers wird sukzessive erhöht,
bis das System \textbf{grenzstabil ist} (endlos mit gleicher Amplitude schwingt).

\begin{minipage}[c]{0.55\columnwidth}
    \begin{tabular}{|c | c | c | c|}
        \toprule
        Regler      & $K_R$                     & $T_N$                 & $T_V$                 \\
        \midrule
        P-Regler    & $0.5 \cdot K_{\rm RES}$   & $-$                   & $-$                   \\
        \midrule
        PI-Regler   & $0.45 \cdot K_{\rm RES}$  & $0.85 \cdot T_{\pi}$  & $-$                   \\
        \midrule
        PID-Regler  & $0.60 \cdot K_{\rm RES}$  & $0.50 \cdot T_{\pi}$  & $0.125 \cdot T_{\pi}$ \\
        \bottomrule
    \end{tabular}
\end{minipage}
\hfill
\begin{minipage}[c]{0.42\columnwidth}
    \begin{center}
        \textbf{\myul{Parameter bestimmen}}
    \end{center}

    \begin{outline}
        \1 Wenn System grenzstabil: Kritisches $K_R$ bestimmen
            \2 $K_{\rm krit} = K_{\rm RES}$
        \1 $T_{\pi}$ Periodendauer der grenzstabilen Schwingung 
    \end{outline}
\end{minipage}

% TODO: Frage: Welche Topologie haben Regler? 



% \subsection{Regler-Einstellung durch Optimierung}